% Edward Hu's Resume
% Created: 7 Jan 2018
% Last Modified: 17 Jul 2018

\documentclass[11pt,oneside]{article}
\usepackage{geometry}
\usepackage[T1]{fontenc}

\pagestyle{empty}
\geometry{letterpaper,tmargin=1in,bmargin=1in,lmargin=1in,rmargin=1in,headheight=0in,headsep=0in,footskip=.3in}

\setlength{\parindent}{0in}
\setlength{\parskip}{0in}
\setlength{\itemsep}{0in}
\setlength{\topsep}{0in}
\setlength{\tabcolsep}{0in}

% Name and contact information
\newcommand{\name}{Edward Hu}
\newcommand{\addr}{Austin, TX 78712}
\newcommand{\phone}{(512) 519-0598}
\newcommand{\email}{bodunhu@utexas.edu}
\newcommand{\website}{https://bdhu.github.io/}


%%%%%%%%%%%%%%%%%%%%%%%%%%%%%%%%%%%%%%%%%%%%%%%%%%%%%%%%%
% New commands and environments

% This defines how the name looks
\newcommand{\bigname}[1]{
	\begin{center}\fontfamily{phv}\selectfont\Huge\scshape#1\end{center}
}

% A ressection is a main section (<H1>Section</H1>)
\newenvironment{ressection}[1]{
	\vspace{4pt}
	{\fontfamily{phv}\selectfont\Large#1}
	\begin{itemize}
	\vspace{3pt}
}{
	\end{itemize}
}

% A resitem is a simple list element in a ressection (first level)
\newcommand{\resitem}[1]{
	\vspace{-4pt}
	\item \begin{flushleft} #1 \end{flushleft}
}

% A ressubitem is a simple list element in anything but a ressection (second level)
\newcommand{\ressubitem}[1]{
	\vspace{-1pt}
	\item \begin{flushleft} #1 \end{flushleft}
}

% A resbigitem is a complex list element for stuff like jobs and education:
%  Arg 1: Name of company or university
%  Arg 2: Location
%  Arg 3: Title and/or date range
\newcommand{\resbigitem}[3]{
	\vspace{-5pt}
	\item
	\textbf{#1}---#2 \\
	\textit{#3}
}

% This is a list that comes with a resbigitem
\newenvironment{ressubsec}[3]{
	\resbigitem{#1}{#2}{#3}
	\vspace{-2pt}
	\begin{itemize}
}{
	\end{itemize}
}

% This is a simple sublist
\newenvironment{reslist}[1]{
	\resitem{\textbf{#1}}
	\vspace{-5pt}
	\begin{itemize}
}{
	\end{itemize}
}



%%%%%%%%%%%%%%%%%%%%%%%%%%%%%%%%%%%%%%%%%%%%%%%%%%%%%%%%%
% Now for the actual document:

\begin{document}

\fontfamily{ppl} \selectfont

% Name with horizontal rule
\bigname{\name}

\vspace{-8pt} \rule{\textwidth}{1pt}

\vspace{-1pt} {\small\itshape \addr \hfill \phone; \email}

\vspace{8 pt}




%%%%%%%%%%%%%%%%%%%%%%%%
\begin{ressection}{Education}

	\begin{ressubsec}{The University of Texas at Austin}{Austin, UT}{Majoring in Computer Science and in Mathematics}
		\ressubitem{GPA: 3.72}
		\ressubitem{Projected Graduation Date: May, 2020}
	\end{ressubsec}

\end{ressection}


%%%%%%%%%%%%%%%%%%%%%%%%
\begin{ressection}{Experience}

	\begin{ressubsec}{UT Computer Science Department}{Austin, UT}{Student Researcher: August 2016--present}
		\ressubitem{Configure Raspberry Pi to cooperate with camera module}
		\ressubitem{Utilize Unix thread to increase algorithm efficiency}
		\ressubitem{Develop toolchain to automate 3-D printing error detection}
		\ressubitem{Implement neural network in Python}
	\end{ressubsec}

	\begin{ressubsec}{Lenovo}{Chengdu, China}{Marketing Intern: May, 2014--July, 2014}
		\ressubitem{Automate marketing research analysis using Python}
		\ressubitem{Participate in the promotion of Thinkpad X1 Carbon}
	\end{ressubsec}

\end{ressection}

\begin{ressection}{Projects}

	\begin{reslist}{PintOS:}
		\ressubitem{Implement shell, threads, system call, virtual memory, and file system to create a bootable operating system}
	\end{reslist}
	
	\begin{reslist}{Parallel K-means algorithm}
		\ressubitem{Using Unix threads and concurrency technique to parallel k-means algorithm to achieve scalability}
	\end{reslist}
	
	\begin{reslist}{File compressor}
		\ressubitem{Using Haffman coding to reduce file size}
	\end{reslist}
\end{ressection}

%%%%%%%%%%%%%%%%%%%%%%%%
\begin{ressection}{Relevant Classes}

	\resitem{Concurrency, Principles of Computer Systems, Computer Organization and Architecture, Algorithms and Complexity, Computational Intelligence in game AI, Data Structure}
\end{ressection}

%%%%%%%%%%%%%%%%%%%%%%%%
\begin{ressection}{Skills}

	\resitem{\textbf{Tools:} Linux (Ubuntu and Debian), UNIX (several variants), Git, Vim}

	\begin{reslist}{Computer Languages:}

		\ressubitem{Proficient in C, C++, Java, Python, \LaTeX, UNIX Shells}

		\ressubitem{Familiar with Swift, Go, x86 assembly}

	\end{reslist}


\end{ressection}


%%%%%%%%%%%%%%%%%%%%%%%%


\end{document}
