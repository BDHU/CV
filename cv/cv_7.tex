%%%%%%%%%%%%%%%%%%%%%%%%%%%%%%%%%%%%%%%%%
% "ModernCV" CV and Cover Letter
% LaTeX Template
% Version 1.3 (29/10/16)
%
% This template has been downloaded from:
% http://www.LaTeXTemplates.com
%
% Original author:
% Xavier Danaux (xdanaux@gmail.com) with modifications by:
% Vel (vel@latextemplates.com)
%
% License:
% CC BY-NC-SA 3.0 (http://creativecommons.org/licenses/by-nc-sa/3.0/)
%
% Important note:
% This template requires the moderncv.cls and .sty files to be in the same 
% directory as this .tex file. These files provide the resume style and themes 
% used for structuring the document.
%
%%%%%%%%%%%%%%%%%%%%%%%%%%%%%%%%%%%%%%%%%

%----------------------------------------------------------------------------------------
%	PACKAGES AND OTHER DOCUMENT CONFIGURATIONS
%----------------------------------------------------------------------------------------

\documentclass[11pt,a4paper,sans]{moderncv} % Font sizes: 10, 11, or 12; paper sizes: a4paper, letterpaper, a5paper, legalpaper, executivepaper or landscape; font families: sans or roman

\moderncvstyle{classic} % CV theme - options include: 'casual' (default), 'classic', 'oldstyle' and 'banking'
\moderncvcolor{blue} % CV color - options include: 'blue' (default), 'orange', 'green', 'red', 'purple', 'grey' and 'black'
\usepackage{multicol}
\usepackage{lipsum} % Used for inserting dummy 'Lorem ipsum' text into the template
\usepackage[hyperref]{}
\usepackage[scale=0.9]{geometry} % Reduce document margins
%\setlength{\hintscolumnwidth}{3cm} % Uncomment to change the width of the dates column
%\setlength{\makecvtitlenamewidth}{10cm} % For the 'classic' style, uncomment to adjust the width of the space allocated to your name

%----------------------------------------------------------------------------------------
%	NAME AND CONTACT INFORMATION SECTION
%----------------------------------------------------------------------------------------

\firstname{ Edward} % Your first name
\familyname{Hu} % Your last name

% All information in this block is optional, comment out any lines you don't need
%\title{Curriculum Vitae}
%\address{116 RKB, IIT Roorkee}{Uttarakhand, India 247667}
\mobile{512 517 0598 }
%\phone{(000) 111 1112}
%\fax{(000) 111 1113}
\email{bodunhu@utexas.edu}
\homepage{bodunhu.com}{www.bodunhu.com} % The first argument is the url for the clickable link, the second argument is the url displayed in the template - this allows special characters to be displayed such as the tilde in this example
\extrainfo{\href{https://github.com/BDHU}{Github: bdhu}}
%\photo[70pt][0.4pt]{pictures/picture} % The first bracket is the picture height, the second is the thickness of the frame around the picture (0pt for no frame)
%\quote{"A witty and playful quotation" - John Smith}

%----------------------------------------------------------------------------------------

\begin{document}

%----------------------------------------------------------------------------------------
%	COVER LETTER
%----------------------------------------------------------------------------------------

% To remove the cover letter, comment out this entire block

%\recipient{HR Department}{Corporation\\123 Pleasant Lane\\12345 City, State} % Letter recipient
%\date{\today} % Letter date
%\opening{Dear Sir or Madam,} % Opening greeting
%\closing{Sincerely yours,} % Closing phrase
%\enclosure[Attached]{curriculum vit\ae{}} % List of enclosed documents
%
%\makelettertitle % Print letter title
%
%\lipsum[1-2] % Dummy text
%\lipsum[4] % Dummy text
%
%\makeletterclosing % Print letter signature
%
%\newpage

%----------------------------------------------------------------------------------------
%	CURRICULUM VITAE
%----------------------------------------------------------------------------------------

\makecvtitle % Print the CV title

%----------------------------------------------------------------------------------------
%	EDUCATION SECTION
%----------------------------------------------------------------------------------------

\section{Education}

\cventry{2016--2020}{B.S. in Computer Science}{The University of Texas at Austin}{Austin TX}{}{}  % Arguments not required can be left empty
\subsection{Related Courses}
\cventry{}{}{}{}{}{Multicore OS, Concurrency, Parrallel Computing, Operating Systems, Networks, AI, Neural Network, Computer Architecture, Algorithm, NLP}
\subsection{Research Interests}
\cventry{}{}{}{}{}{Operating Systems, GPUs, Heterogeneous Systems, Distributed Computing \& Systems, ML Systems}
%\cventry{2016}{D.A.V. Public School, Bistupur}{\newline Mathematics and Computer Science}{Class 12}{}{GPA -- 94.00\%}  % 
%\cventry{2014}{A.D.L.S. Sunshine School, Sakchi}{\newline Mathematics and Computer Science}{Class 10}{}{GPA -- 95.20\%}  %
%\section{%Masters Thesis}

%\cvitem{Title}{\emph{Money Is The Root Of All Evil -- Or Is It?}}
%\cvitem{Supervisors}{Professor James Smith \& Associate Professor Jane Smith}
%\cvitem{Description}{This thesis explored the idea that money has been the cause of untold anguish and suffering in the world. I found that it has, in fact, not.}

%----------------------------------------------------------------------------------------
%	WORK EXPERIENCE SECTION
%----------------------------------------------------------------------------------------

\section{Experience}

%\subsection{%Vocational}
% \cvitem{2018--present}{\textbf{Recobell} \textit{Seoul Korea, https://www.recobell.com/   } -- Data Scientist}

\cventry{2018 -- present}{UTCS System Research Lab}{Austin TX}{Researcher}{\textsc{Advisor:  Christopher Rossbach}}{
The use of graphics processing units (GPUs) has become
increasingly popular in accelerated computing in industry
in recent years. Existing benchmark suites such as Rodinia were designed to better understand heterogeneous systems. But these tools failed to keep up with the rapid evolution of GPU. We develop a new benchmark suite to better demonstrate the diversity of GPU workloads, applications, and GPU resource usage, with emphasis on neural network workload characterizations and new CUDA features.}
%commented out cotents
%\begin{itemize}
% \item Implemented various image segmentation models like U-Net, SegNet, etc.
% \item Currently working on satellite imagery.
% \item Created tests and automation scripts.
% \end{itemize}


\cventry{2016 -- 2017}{UTCS AI Research Lab}{Austin TX}{Student Researcher}{\textsc{Advisor:  Cem Tutum}}{
3-D printing has introduced many possibilities into the manufacturing industry. However, detecting object defection in real time requires expensive equipment inaccessible to many. With a team of four, we classified the types of error presented in 3-D printing. Because there were no previous work for reference, we delivered a framework to automate the process of data collection by alternating Gcode instructions, which facilitated the training of neural networks used to classify error type present in collected data.
}

\renewcommand{\listitemsymbol}{\textcolor{black}{-~}}
\cventry{June -- Aug 2018}{H3C}{Chengdu, China}{Software Engineering Intern}{}{
\begin{itemize}
    \item Reconstructed mjpg-streamer to increase jpg streams efficiency in Linux by ten percent.
    \item Participated in the integration of Apollo autonomous vehicle driving system into H3C cloud platform (in Go). 
\end{itemize}
}

\cventry{June -- Aug 2017}{Wisesoft}{Chengdu, China}{Junior Software Engineer}{}{
\begin{itemize}
    \item Implemented the neural network module (in Tensorflow) to process audio files using FFT(Fourier transformation). It is used in a production level system used to classify communication in air traffic control system.
    \item Constructed a tool set to automatically prune neural network models and organize training data files. Deployed into production systems.
\end{itemize}
}

%comment out
% \begin{itemize}
% \item Implemented research papers based on \textsc{Finance and Quant trading}.
% \item Wrote a \textsc{MLP classifier} to generate trade signals that increased the accuracy by 1.02\%.
% \item Implemented LSTM for stock price prediction.
% \end{itemize}

%------------------------------------------------

%\cventry{2011--2012}{Summer Intern}{\textsc{Lehman Brothers}}{Los Angeles}{}{Rated "truly distinctive" for Analytical Skills and Teamwork.}

%------------------------------------------------

\section{Projects}

\cventry{Jan 2019}{GUPS on GPU}{\textit{Random Memory Access on GPU}}{}{}{
\begin{itemize}
\item Constructed random memory access program to stress the dram capability of GPU with several optimizations. 
\end{itemize}}

\cventry{April 2019}{FDTD in Cuda Graph}{\textit{Reimplementation in Cuda Graph}}{}{}{
\begin{itemize}
\item Rewrote part of fdtd(Finite-difference time-domain method) using Cuda graph to test new Cuda features.
\end{itemize}}

\cventry{April 2019}{Multicore Operating System Implementation }{a capability-based research OS by ETH Zurich}{}{}{
\begin{itemize}
\item Implemented physical memory management and process spawning based on Barrelfish kernel.
\item Implemented messege passing in LMP and RPC for processes to communicate in a multicore system. 
\end{itemize}}

\cventry{Spring 2018}{Kmeans in CUDA MPI }{Kmeans in NUMA}{}{}{
\begin{itemize}
\item Developed kmeans with focus on MPI communication on TACC super computer to test program scalability.
\end{itemize}}

\cventry{Spring 2018}{Thread Pool in C }{a thread pool implemented in C}{}{}{
\begin{itemize}
\item Designed a thread pool with tasks mapped upon pthreads to simulate the design of Go routines.
\end{itemize}}

% \cventry{Spring 2019}{Neural Arithmetic Logic Units \href{https://github.com/sinAshish/NALU-Keras}{[Code]}}{\textit{Self-motivated}}{}{}{
% \begin{itemize}
% \item Implemented the paper \textit{Neural Arithmetic Logic Units} by Trask et. al. in Keras.
% \item The code does \textit{arithmetic operations} only, for now.
% \end{itemize}}
 
% \cventry{Aug 2018-- Oct 2018}{MURA (musculoskeletal radiographs) X-Ray Classification \href{https://github.com/sinAshish/Stanford-MURA}{[Code]}}{\textit{Self-motivated}}{}{}{
% \begin{itemize}
% \item Implemented the original paper to reproduce the results.
% \item Optimized  the hyperparameters to beat the baseline score.
% \item Used pretrained models like DenseNet50, ResNet169 among others with attention for classification.
% \end{itemize}}
 
% \cventry{Oct 2018}{Dynamic Memory Networks Plus \href{https://github.com/sinAshish/dmn-plus}{[Code]}}{\textit{Self-motivated}}{}{}{
% \begin{itemize}
% \item Implemented the paper \textit{Dynamic Memory Networks for Visual and Textual Question Answering} in PyTorch.
% \item Model trained on BaBI dataset.
% \end{itemize}}

% \cventry{May 2018--  Aug 2018}{Simplifying Rough Sketches Using Deep Learning \href{https://github.com/sinAshish/Rough-Sketch-Simplification-Using-FCNN}{[Code]}}{\textit{Self-motivated}}{}{}{
% \begin{itemize}
% \item Implemented the paper \textit{Learning to Simplify: Fully Convolutional Networks for Rough Sketch Cleanup} by Simo-Serra et. al in PyTorch.
% \item Implemented an Encoder-Decoder architecture.
% \end{itemize}}

% \cventry{ Nov 2018}{Quora Insincere Question Classification}{\textit{Self-motivated}}{}{}{
% \begin{itemize}
% \item Implemented a CNN-LSTM architecture with attention to detect toxic content in online media. 
% \item Achieved an F1-Score of 0.73.
% \end{itemize}}

%\cventry{ Mar 2017}{Gaana (Songs) Downloader \href{https://github.com/sinAshish/Gaana-Downloader}{[Code]}}{\textit{Self-motivated}}{}{}{
%\begin{itemize}
%\item A script to download all the songs from \href{https://www.gaana.com}{gaana.com} by just entering the album name.
%\end{itemize}}
%
%----------------------------------------------------------------------------------------
%	AWARDS SECTION
%----------------------------------------------------------------------------------------

% \section{Achievements}

% \cventry{July 2018}{Humpback Whale Identification Challenge}{\textit{Kaggle}}{}{}{
% \begin{itemize}
% \item Ranked 64 among 528 teams( TOP 13\%) in the Whale Identification Challenge.
% \item Used Transfer Learning with ResNet50.
% \end{itemize}}

% \cventry{Oct 2018}{TGS Salt Identification Challenge}{\textit{Kaggle}}{}{}{
% \begin{itemize}
% \item Ranked 589 among 2754 teams( TOP 22\%) in the task to segment salt deposits beneath Earth's surface.
% \item Used UNet with ResNet50 as encoder.
% \end{itemize}}

% \cventry{Aug 2018}{Home Credit Default Risk}{\textit{Kaggle}}{}{}{
% \begin{itemize}
% \item Ranked 1685 among 7198 teams( TOP 24\%) in the Home Credit Default Risk Challenge.
% \item Used LightGBM and XGBoost with weighted ensemble.
% \end{itemize}}

% \cventry{Nov 2018}{Human Protein Atlas Image Classification}{\textit{Kaggle}}{}{}{
% \begin{itemize}
% \item Ranked in the TOP 17\%(now) in the multiclass multilabel protein classification challenge.
% \item Used ResNet50 as base model.
% \end{itemize}}


% \cventry{Oct 2018}{Quick Draw! Doodle Recognition Challenge}{\textit{Kaggle}}{}{}{
% \begin{itemize}
% \item Ranked in the TOP 29\% in the Doodle Recognition Challenge.
% \item Trained an ImageNet model from scratch, by constructing images from stokes.
% \end{itemize}}




%----------------------------------------------------------------------------------------
%	COMPUTER SKILLS SECTION
%----------------------------------------------------------------------------------------

\section{Skills}

\cvitem{Languages}{Python, C/C++, Java, Go, Rust}
\cvitem{Frameworks}{OpenMP, MPI, PyTorch, Tensorflow, Cuda, Jekyll}
% \cvitem{WebD}{HTML/CSS, JavaScript, Jekyll}
% \cvitem{Utilities}{Anaconda, Git, Sublime Text, Jupyter Notebook}
% \cvitem{Communication}{English(SRW), Hindi(SRW), Japanese(SO)}
%----------------------------------------------------------------------------------------
%	COMMUNICATION SKILLS SECTION
%----------------------------------------------------------------------------------------

% \section{Relevant Courses}

% \cvitem{Online}{Convolutional Neural Networks for Visual Recognition, Deep Learning for NLP, Deep Learning.Ai Specialization, Introduction to Machine Learning, Introduction to Statistics and Probability}
% \cvitem{Classroom}{ Linear Algebra, Differential Calculus, Differential Equations, Economics, Marketing Research, Environmental Economics}

% \section{Extra Curriculars}

% \cventry{Apr 2018}{Vision and Language Group}{\textit{Executive Member}}{}{}{ The group aims to foster Deep Learning research among students by conducting discussions and implementations on various Research Papers in the field of Computer Vision and NLP. I implemented a paper on DC-GAN and Neural Style Transfer.
% }

% \cventry{Oct 2017}{Enactus IIT Roorkee Chapter}{\textit{Executive Member}}{}{}{ The Campus group works towards Socail Entreprenership. I was a part of the Kaagaz Project.
% }

% \cventry{Jan 2018}{Academic Reinforcement Program}{\textit{Teaching Assistant}}{}{}{ Taught General Chemistry(CYN-006) to a batch of 86 students.
% }
% \cventry{Jul 2018}{Academic Reinforcement Program}{\textit{Teaching Assistant}}{}{}{ Taught Intro to Computer Programming in C++(MTN-103) to a batch of 80 students.
% }

% \cventry{Feb 2018}{Sangram IIT Roorkee}{\textit{Web Developer}}{}{}{ Developed the website for Sangram, IIT Roorkee, the official annual Sports fest of IIT Roorkee.
% }

% \cventry{Dec 2015}{Quizense}{\textit{Founder}}{}{}{ Started a start-up along with 3 others that aimed to conduct Quizzes for various schools and fests.
% }
%----------------------------------------------------------------------------------------
%	INTERESTS SECTION
%----------------------------------------------------------------------------------------

%\section{Interests}

%\renewcommand{\listitemsymbol}{-~} % Changes the symbol used for lists

%\cvlistdoubleitem{Cycling}{Hiking}
%\cvlistdoubleitem{Sketching}{Gaming}
%\cvlistitem{Quizzing}
%----------------------------------------------------------------------------------------

% \section{References}

% \begin{multicols}{2}
% \cventry{}{K.S Suresh}{\newline Assistant Professor}{\newline Metallurgical and Materials Engg., IIT Roorkee}{\newline suresfmt@iitr.ac.in}{}
% \columnbreak
% \cventry{}{Anu Chandra}{\newline CEO}{\newline Ryelore AI}{\newline anu@ryelore.com}{ }
% \end{multicols}
% %\cventry{}{Arpit Gupta}{\newline VP Engineering}{\newline Antriex IT Services}{\newline arpit.gupta@antmex.com}{}

% %\cventry{}{B.S.S Daniel}{\newline Professor}{\newline Metallurgical and Materials Engg., IIT Roorkee}{\newline s4danfmt@iitr.ac.in}{ }
\end{document}