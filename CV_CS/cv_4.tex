%----------------------------------------------------------------------------------------
%	PACKAGES AND OTHER DOCUMENT CONFIGURATIONS
%----------------------------------------------------------------------------------------

\documentclass{resume} % Use the custom resume.cls style

\usepackage[left=0.75in,top=0.6in,right=0.75in,bottom=0.6in]{geometry} % Document margins

\name{Edward Hu} % Your name
\address{2819 Deeds Road \\ Houston, TX 78705} % Your address
%\address{123 Pleasant Lane \\ City, State 12345} % Your secondary addess (optional)
\address{https://github.com/BDHU}
\address{(512)~$\cdot$~517~$\cdot$~0598 \\ bodunhu@utexas.edu} % Your phone number and email

\begin{document}

%----------------------------------------------------------------------------------------
%	EDUCATION SECTION
%----------------------------------------------------------------------------------------

\begin{rSection}{Education}

{\bf University of Texas at Austin} \hfill {\em December 2016 - May 2020} \\ 
B.S. in Computer Science \& Mathematics \\
%Minor in Linguistics \smallskip \\
GPA: 3.72

\end{rSection}

%----------------------------------------------------------------------------------------
%	WORK EXPERIENCE SECTION
%----------------------------------------------------------------------------------------

\begin{rSection}{Experience}

\begin{rSubsection}{University of Texas at Austin}{August 2016 - Present}{Student Researcher}{Austin, TX}
\item Used OpenCV to detect the defects generated during 3D printing and halt the process if necessary.
\item Construct artificially neural networks with multiple layers.
\item Use gradient descent and genetic algorithms to optimze the performance of the ANN.
\item Utilized numpy and matlibplot for the optimization and graphing tasks.
\item Use Python to modify the Gcode file used to guide the 3D printing process.
\item Use optimization methods to find the optimal solution to cut 3D printed object to reduce support structure required.

\end{rSubsection}
\begin{rSubsection}{Lenovo}{May 2014 - July 2014}{Marketing Intern}{Chendu, China}
\item Participation in promotion for ThinkPad X1 Carbon
\item Help design questions for interviewing interns in colleges.
\end{rSubsection}

\end{rSection}

%----------------------------------------------------------------------------------------
%	Projects
%----------------------------------------------------------------------------------------
\begin{rSection}{Technical Experiences}
\begin{rSubsection}{Projects}{}{}{}
\item \textit{Minesweeper Optimization}: Used NEAT framework to implement ANN to improve the efficiency of minesweepers.
\item \textit{Character Recognizer}: Implement a two-layer neural network in Python to improve its accuracy on predicting the hand-written digits.
\item \textit{File Compressor}: Used huffman coding method to compress a file, reduce the size of the file, and restore the compressed data with Java.

\end{rSubsection}

\begin{rSubsection}{Extra curriculum}{}{}{}
\item IEEE Robotics \& Automation Society
\item UT Solar Vehicles Team
\end{rSubsection}

\begin{rSubsection}{Curriculum}{}{}{}
\item Data Structure, Intro to Computer Architecture, Intro to Computer Systems, Computational Intelligence in game AI, Practical Linear Algebra for Computer Science
\end{rSubsection}

\end{rSection}



%----------------------------------------------------------------------------------------
%	TECHNICAL STRENGTHS SECTION
%----------------------------------------------------------------------------------------

\begin{rSection}{Technical Strengths}

\begin{tabular}{ @{} >{\bfseries}l @{\hspace{6ex}} l }
Computer Languages & Java, C/C++, Python, Swift, \LaTeX, Matlab \\
Tools & Linux, Git, Vim, GCC, Docker, Intellij, Xcode \\
Languages & Chinese, English
\end{tabular}

\end{rSection}

%----------------------------------------------------------------------------------------
%	EXAMPLE SECTION
%----------------------------------------------------------------------------------------

%\begin{rSection}{Section Name}

%Section content\ldots

%\end{rSection}

%----------------------------------------------------------------------------------------

\end{document}
